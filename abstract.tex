

\begin{abstract}
We evaluate the use of recurrent neural networks (RNNs) for the task of detecting exoplanet transit signals in light curves of stars. A bidirectional GRU RNN is trained to classify individual data points as being part of a transit signal or not, and its outputs are used in different ways to define candidate detections. As opposed to commonly used approaches, the proposed method does not require inputs to be detrended, nor does it require a search through grids of parameter values which specify the shape of the signal. In experiments using simulated data, we address the problem of data gaps, data imbalances, interpretability, applicability, and compare our RNN-based detection methods against box-fitting algorithms in imitated real-world problems. In the task of retrieving 2500 monotransits from 5000 simulated light curves, the RNN outperformed our baseline both in terms of average precision (AP), i.e. 0.43 versus 0.35, and computational requirements, i.e. a few minutes versus several hours. In the task of retrieving repeating transit signals from 2500 planets in 5000 light curves, the box-fitting algorithm seemed to benefit more from the periodicity of the signals than our best RNN-based algorithm. The box-fitting algorithm outperformed our RNN-based algorithm in terms of AP, i.e. 0.66 versus 0.59, but not in terms of computational requirements. In both cases, however, we found the methods to detect planets that the other was not able to detect, indicating that these methods worked complementary to each other. This result suggests that our novel RNN-based approach could open the door to detecting exoplanets previously missed by conventional methods, may it be applied to real-world data in the future. \\\\
\noindent\begin{footnotesize}
\textbf{Code:} \url{https://github.com/ykerus/transit-detection-rnn}
\end{footnotesize}
\end{abstract}