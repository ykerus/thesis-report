

\begin{abstract}
We evaluate the use of recurrent neural networks (RNNs) for the task of detecting exoplanet transit signals in the light curves of stars. A bidirectional GRU RNN is trained to classify individual data points as being part of a transit signal or not, and its outputs are used to define candidate detections. This approach is novel as it is the first RNN-based approach to transit detection, and it does not require inputs to be detrended, nor does it require a grid-search of parameter values for specifying the signal’s shape during search. We address the problems of data gaps, data imbalances, interpretability, applicability, and compare our RNN-based detection methods against the baseline, i.e. box-fitting algorithms, on simulated data. 

The performance is evaluated on the task of retrieving monotransits, within data samples that include only a single transit, and retrieving repeating transits, within data samples that include multiple transits. On both tasks, the RNN-based algorithm was computationally faster than the baseline, in some cases up to two orders of magnitude. In the monotransit task, the RNN outperformed the baseline in terms of average precision (AP); i.e. 0.43 AP versus 0.35 AP. In the repeating transit task however, the box-fitting algorithm benefited more from the periodicity of the signals and outperformed our RNN-based algorithm; i.e. 0.66 AP versus 0.59 AP. 

As demonstrated in this thesis, these methods can complement each other; with one detecting a transit signal where the other could not. This suggests that our RNN-based approach could open the door to detecting exoplanets previously missed by conventional methods, should it be applied to real-world data in the future. \\\\
\noindent\begin{small}
\textbf{Code:} \url{https://github.com/ykerus/transit-detection-rnn}
\end{small}
\end{abstract}