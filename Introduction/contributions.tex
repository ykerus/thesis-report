
\section{Summary of contributions}
\label{sec:contributions}

This work reports the first attempt in literature of using RNNs for the task of transit detection. The research presented revolves around the question of whether and how RNNs could be applied to compete with conventional transit detection algorithms. We provide a detailed analysis of what preprocessing steps could be considered when applying this network to light curves to search for signals. Successes and failure cases of our method are identified, and the general performance is evaluated in comparison to existing approaches. Our approach does not require detrending of input light curves as opposed to commonly used methods. During search, the RNN does not iterate over transit durations, depths or other parameters that define the signal's shape, since the model is implicitly trained to recognize these features prior to application.

In short, we found the RNN to show most potential in the task of monotransit detection, because it does not rely on periodicities of signals and it can be applied highly efficiently. It outperformed the box-fitting algorithm of \cite{foreman2016population} in the task of retrieving single transit events in simulated light curves. In the case of multiple transit signals in a single light curve, we found the BLS algorithm to benefit more from the periodicity of the signal, as it outperformed our RNN-based algorithms. In both cases however, the algorithms tested were able to detect signals that the other one was not able to find. In other words, the methods performed complementary. This result suggests that an RNN-based transit detection method could open the door to detecting exoplanets that would otherwise be overlooked by conventional algorithms.