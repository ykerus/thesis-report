
\section{Contributions}

We report the first attempt in literature of using recurrent neural networks (RNNs) for the task of detecting transit signals in light curves. This particular network can handle arbitrary input sizes and can provide predictions at every time step of the light curve, making it ideal for detection as it allows us to locate potential signals in time. Furthermore, during the training phase, different transit shapes can be fed to the network and the network is trained to distinguish these from complex background patterns. For the detection of individual transit events, the light curve thus does not need to be detrended and only has to be passed through the network once, without iterating over different transit durations, depths and shapes.

In this thesis, we investigate whether an RNN-based approach to detecting transit signals is viable and competitive as compared to more classical approaches. Successes and shortcomings are identified using simulated data, and potential extensions to this approach are explored. These extensions include the use of confidence estimation over the RNN predictions for better monotransit detection and the use of hidden representations of transit signals to disambiguate multiple repeating signals in a single light curve.

[TODO: list findings: focus on monotransits]

This work can be used as reference (i) for astronomers to how AI can help solve problems in possibly previous unforeseen ways, and (ii) for AI researchers to how a well-established network design can be used in new ways for complex and important problems outside the scope of planet Earth.