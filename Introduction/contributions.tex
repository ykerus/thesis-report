
\section{Contributions}

We report the first attempt in literature of using recurrent neural networks (RNNs) for the task of detecting transit signals in light curves. This particular network can handle arbitrary input sizes and can provide predictions at every time step of the light curve. The properties are ideal for detection as they allows us to locate potential signals in time for any given light curve. Furthermore, during the training phase, different transit shapes can be fed to the network. When searching for new transit signals, this prior knowledge is taken into account by the network, so there is no need to iterate over transit durations, depths and shapes anymore. Furthermore, the network can be trained to distinguish these signals from complex background patterns without requiring the input to be detrended.

In this thesis, we investigate whether an RNN-based approach to detecting transit signals is viable and competitive as compared to more classical approaches. Successes and shortcomings are identified using simulated data, and potential extensions to this approach are explored. These extensions include the use of confidence estimation over the RNN predictions for better monotransit detection and the use of hidden representations of transit signals to disambiguate multiple repeating signals in a single light curve.

\todo{list findings and highlight potential for monotransit detection}

This work can be used as reference for astronomers to how AI can help solve problems in possibly previous unforeseen ways, and for AI researchers to how a well-established network design can be used in new ways for complex and important problems outside the scope of planet Earth.