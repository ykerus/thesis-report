
\section{Problem}

In short, the problem addressed in this thesis is that of detecting dips in stellar brightness that could indicate the presence of an transiting exoplanet. The signal may repeat itself due to the periodicity of the planet's orbit, but it may also only occur once if the observation span is shorter than the orbital period of the planet. These candidate transit signals are searched for in data we refer to as \textit{light curves}, which describe the observed brightness, or flux, of stars over time.  However, while the physics of transits is well-understood, the detection of their signatures in light curves is hindered by stellar and spacecraft induced noise. Furthermore, transit signal shapes vary, and the chance that any given light curve contains detectable transit signals is low, especially in the case of small, Earth-like planets. For a full description of the challenges and their origins, see Section \ref{sec:challenges}. 

In the problem of detecting transit signals in light curves, we aim to find an algorithm that is capable of specifying a reference time $t_0$ and the period $P$ (if possible) of potential exoplanet candidates, given an input light curve. The reference time $t_0$, also referred to as epoch, is generally the mid-transit time of the first transit event in the light curve. These parameters are the minimum requirement to allow follow-up research to retrieve the signal that was detected.