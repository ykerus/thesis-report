
\section{Problem}

Different approaches to exoplanet discovery exist. This work focuses on the transit method, the most successful method in terms of number of discoveries. The transit method relies on the rare event that a planet moves in front of its host star in our line of sight, i.e. the so-called ``transit''. Transits express themselves in the form of periodic dips in the observed brightness of a star over time. The detection of these dips in stellar data is therefore essential for the discovery of new exoplanets. 

The observed brightness, or flux, of a star over time is referred to as the star's ``light curve''. Generally, transit signals have a duration in the order of hours and light curves have a time span in the order of days or months. A transit signal may repeat itself if the orbital period of the exoplanet is shorter than the time span of the light curve.
In case the orbital period longer, a transit signal may occur only once, in which case we speak of a ``monotransit''. 
Moreover, multiple planets might orbit the same star, in which case we can have multiple transit signals from different planets in the same light curve.

The physics of transits is well-understood, but the detection of their signatures in light curves is hindered by stellar and spacecraft induced noise. 
In addition, transit signals may vary in depth, duration and general shape depending on stellar and planetary parameters.
Furthermore, transit signals are in general weak and sparse, so the chance of any given light curve to contain detectable transit signals is low. A detailed description of the challenges and their origins is given in Section \ref{sec:challenges}.

The problem we address is thus the detection of dips in light curves of stars that could indicate the presence of an orbiting exoplanet. In literature, however, some ambiguity exists in the naming of different steps in the process of discovering transiting exoplanets. In the detection step, i.e. the focus of this thesis, we search for potential signals in a light curve, and provide parameters such as the epoch $t_0$ and orbital period $P$ of the detected candidates. The epoch is a reference time, e.g. the time of the first transit event in the light curve, which together with $P$ allows for retrieving the signal that was found. Detection should not be confused with identification or vetting, the step in which the detected signals are classified as false positive, planet candidate, or candidate of another type of object.
