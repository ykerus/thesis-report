
\section{Proposed method}
\label{sec:proposed}

We evaluate the use of RNNs for the task transit signal detection. As opposed to CNNs, RNNs can (i) handle arbitrary input sizes and (ii) provide an output at every time step. We make use of property (i) by training the network on light curve segments and applying it to full-length light curves for detection. We make use of property (ii) by training the model to identify exactly those time steps that are part of a transit signal. In other words, the RNN is trained to classify each data point as signal or non-signal.

An important factor in the network's performance is the way in which the input data is preprocessed. Therefore, the question arises which preprocessing steps can or should be considered when using RNNs for this task, and how they affect the model’s performance. For example, to address the problem of non-uniform time intervals between measurements, we explore the use of a generative RNN, which is trained to fill in data gaps as it processes through a light curve. 

In experiments using simulated data, we assess whether an RNN-based detection algorithm is a viable and competitive alternative to existing approaches. In the task of monotransit detection, we investigate the question of whether the outputs of the RNN can be made more intuitive, in order to set better thresholds and reduce the amount of false detections. To do so, we extend the network to produce confidence outputs in addition to the standard outputs at each time step, as proposed by \cite{devries2018learning}. The confidence outputs are supposed to indicate the confidence over the corresponding standard outputs at each time step.

In the case of detecting repeating signals, additional steps are necessary to determine the periodicity. Two algorithms are compared in this work, the first of which is inspired by the work by \cite{pearson2018searching}. Similar to \cite{pearson2018searching}, we will refer to the network outputs over time as probability time series (PTS). The first algorithm relies on peaks in the PTS, which indicate the potential detections of individual transit events. It uses the average distance between subsequent peaks as a period estimate of the signal. In the search for multiple planets, one could apply this method iteratively, each time removing the most prominent peaks from the PTS. However, ambiguities may arise if it is not clear which peaks in the PTS belong to which signal. To address this problem, we explore the use of learned signal representations by the RNN, which could help resolve such ambiguities and allow for simultaneous detection of multiple different signals at once.  The second algorithm is similar to many algorithms in that it folds the PTS over a set of trial periods. For each trial period, a score is given based on the aggregated values within the folded PTS, and the best scoring value for the period is used as a period estimate.

