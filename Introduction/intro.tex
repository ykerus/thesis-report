
The discovery of exoplanets is the first step in answering a broad range of fundamental questions in astronomy. Is our solar system one of a kind? Is there life elsewhere in the universe? Exoplanets, short for extrasolar planets, are planets orbiting stars other than our Sun. It was only after the 90's that scientists became certain of their existence and nowadays more than 4000 examples of exoplanets are known\footnote{\url{https://exoplanetarchive.ipac.caltech.edu/docs/counts_detail.html}}. Several approaches to exoplanet discovery exist, the most successful of which is the transit method. This method relies on the rare event that a planet moves in front of its host star with respect to our line of sight, a so-called ``transit''. Such events occasionally occur for Mercury and Venus. Transits express themselves in the form of periodic dips in the observed brightness of a star over time.  The detection of these dips in stellar data is therefore essential for the discovery of new exoplanets. 

In the following sections, the problem of detecting transit signals is described in further detail, as well as the motivation to approach this problem from different perspectives. Limitations of common approaches are discussed, and we describe how the solution to the problem might benefit from artificial intelligence (AI). The contributions and outline of this work are given in the last sections of this chapter.