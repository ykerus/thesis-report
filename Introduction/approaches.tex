
\section{Common approaches and limitations}
\label{sec:approaches}

Most commonly used in the task of transit detection is the Box Least Squares (BLS) algorithm \citep{kovacs2002box}. This algorithm uses a box-shaped model of a strictly periodic signal. In order to search for transits, the model is fitted to the data at different signal durations, periods and epochs. Variants to this algorithm exist, for example Transit Least Squares (TLS) \citep{hippke2019optimized}, or Sparse Box Least Squares (SBLS) \citep{panahi2021sparse}.

One limitation of these methods is their requirement of a detrended input light curve. Detrending is the process of removing irrelevant time-dependent noise from the light curve. In the process, however, information is altered at short time scales, so the signal that we wish to find may get reduced or altered \citep{hippke2019wotan}.  Alternatives have therefore been proposed, for example simultaneously modeling the background while searching for transit signals \cite{foreman2015systematic}.

However, what these methods still have in common is their brute-force approach to searching for signals. They rely on iterating through trial configurations of parameters that define the signal's shape and timing. In particular for transit models more realistic that the box-function, this iteration over potential signals can be costly. In addition, in case the search is directed towards multiple planets in the same light curve, this process has to be repeated several times, each time masking the previous signal that was found. Furthermore, the model configurations to be iterated over are generally specified by the user, which might cause the search to be biased towards specific signal shapes.

An approach based on AI may overcome several of these limitations, as shown by \cite{pearson2018searching}. In their work, the use of a one-dimensional convolutional neural network (CNN) is proposed for the task of transit detection. On the one hand, neural network outputs can be more difficult to interpret than outputs of box-fitting algorithms, because a neural network generally has complex hidden dynamics which make it function like a ``black box''. On the other hand, they come with several benefits. The CNN does not iterate over transit shapes, and detrending the input light curve is not necessary. In the proposed method, however, the CNN is only capable of providing a single output for an input light curve segment of fixed size. This means that it cannot directly be applied to a full-length light curve of arbitrary size, and we need to take additional steps to determine the timing of potential signals. One option is to apply the CNN to overlapping segments in the light curve to obtain outputs at each time step. However, this approach makes the CNN less intuitive and efficient in its use for detection. 

In the task of transit signal detection, the use of recurrent neural networks (RNNs) remains unexplored. The question therefore remains whether RNNs can provide an efficient and competitive alternative to classical transit detection methods, and improve over CNNs in terms of their applicability. Additionally, we raise the question of how the outputs of RNNs can be made more intuitive for the task of detection. 

