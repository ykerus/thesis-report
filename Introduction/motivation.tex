
\section{Motivation}

Within the course of a few decades, the status of exoplanets has changed from hypothetical to common. The discovery of exoplanets has greatly enriched our view on the universe, as with each new discovery we can better answer questions regarding, for example, the distribution of exoplanet compositions or planetary system dynamics and evolution. The search for a second Earth is still going, and with that the search for life. For these reasons, research is always pushed to find undiscovered planets.

Existing approaches to exoplanet discovery each come with their own benefits and drawbacks (see Section \ref{sec:disc_methods}). This thesis focuses only on the transit method, which is motivated by several reasons. Firstly, the problem to solve is relatively straightforward, namely detecting dips in brightness over time. Secondly, plenty of data is available for this method, because the only thing to be measured is the brightness of stars over time, which can be done at a single wavelength interval for many stars simultaneously. Lastly, even though the method is straightforward, a transit signal carries a wealth of information about the planetary system at hand. For example, the depth of the signal is related to both the size of the planet relative to its host star and several orbital parameters such as the inclination of its orbital plane. The periodicity of the signal directly tells us the orbital period of the planet, although even the duration of a single transit signal can already constrain the orbital period. More information about the transit method and what we can learn from it is given in Section \ref{sec:transit_method}. 

Before we can determine all these parameters, however, the signal need to be detected. Several algorithms exist to search for transit signals in light curves, but as each algorithm can differ, so can their returned detections. One algorithm may miss a planet that another algorithm finds and vice versa. For this reason, we aim to develop a new transit detection algorithm which works substantially different from existing methods, that is capable of providing complementary detections.


