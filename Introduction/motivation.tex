
\section{Motivation}

Within the course of a few decades, the status of exoplanets has changed from hypothetical to common. The discovery of exoplanets has greatly enriched our view on the universe, as with each new discovery we can better answer questions regarding, for example, the distribution of exoplanet compositions or planetary system dynamics and evolution. Nevertheless, though several candidates have been identified, the search for a second Earth is still going, and with that the search for life. The general motivation to search for more exoplanets is thereby clarified. 

However, as mentioned, several approaches to exoplanet discovery exist. Each method has its own benefits and drawbacks, for which we refer the reader to Section \ref{sec:disc_methods}. This thesis focuses only on the transit method, which is motivated by several reasons. The first is that the problem to solve is relatively straightforward, i.e. detecting dips in brightness over time. Second, plenty of data is available, because the only thing to be measured is the brightness of stars over time, which can be done at a single wavelength interval for many stars simultaneously. Last, a transit signal carries a wealth of information about the planetary system at hand. For example, the depth of the signal is related to both the size of the planet compared to it's host star and several orbital parameters such as the inclination, the angle of the orbital plane orbit relative to the observer. The periodicity of the signal directly tells us the orbital period of the planet, although the duration of a single transit signal also constrains this parameter. More information about the transit method and what we can learn from it is given in Section \ref{sec:transit_method}. 

Before we can determine all these parameters, however, the signal need to be detected. Several algorithms exist to search for transit signals in light curves, but as each algorithm can differ, so can their returned detections. One algorithm may miss a planet that an other algorithm finds and vice versa. For this reason, in this thesis we aim to develop a new transit detection algorithm, which works substantially different from existing methods.


