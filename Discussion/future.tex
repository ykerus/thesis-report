
\section{Future directions}

A clear direction for the future of the RNN-based detection method is to be applied to real-world data. Especially in the task of monotransit detection, we expect the RNN to lead to interesting new results. In our work, a general search for transits was evaluated, i.e. transits of varying depths and varying shapes. However, the training of the network can also be tuned for the search for specific kinds of transit signals. For example, the training data can be injected with transits from exoplanets with orbiting ``exomoons''.  In this approach, the network can be trained to only detect these special transits, while ignoring the ``normal'' transit signals. Similarly, one could inject the training data with transits only from disintegrating rocky exoplanets (DREs, \cite{rappaport2012possible}), or transit signals from other particularly interesting objects.

To address several of the limitations described in the previous section, future research could also aim at improving the detection method. For example, it would be highly favorable if the network is able to work with periodicities of potential signals, such that the network directly outputs $P$ and $t_0$ and can be trained end-to-end. A considerable portion of this project was devoted to finding a solution for this, without success.  Other directions include the
estimation of more parameters in the output of the network, e.g. the depth or duration of the signal, and estimations of their uncertainties. Further research could also evaluate the effect of EB signals on the performance of the RNN. For example, either by considering EB signals as noise, or by including them in the problem as an additional class of signal.

Lastly, we found the problem of transit detection to not always be defined clearly or consistently in literature.  Therefore we believe it would be worth constructing a benchmark data set with a clearly defined task, specifically for the task of transit signal detection. Not only would this make the comparison between different papers and methods easier, it may also motivate more AI researchers to tackle this problem from different perspectives.

% possible extension to other domains, e.g. biological processes. require short signal, discrete time steps/measurements, noisy background, (periodicity)