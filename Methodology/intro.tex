
The transit detection algorithm consists of two main components: the RNN and the method to determine the period $P$ and epoch $t_0$ for candidate transit signals using the RNN outputs. In the first sections of this chapter, we describe the basis network, its training, and the methods we use to determine $P$ and $t_0$. Sections \ref{sec:rnn_gen} to \ref{sec:rnn_repr} describe how the basis RNN is extended to allow for filling missing values in light curves, confidence estimation over the RNN outputs, and providing learned representations of data patterns and signals in addition to its standard outputs. The last section of this chapter describes which data are used for the experiments.