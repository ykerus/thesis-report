
The network is used to model $p(T|X)$, where $T$ represents the targets $\{t_j\}_{j=1}^N$ for each data point $\{x_j\}_{j=1}^N$ in a given light curve segment $X$. We assume the network inputs to be preprocessed such that the flux values $x_j$ can take any value, i.e. $x_j \in \mathbb{R}$. A target $t_j$ is 1 if $x_j$ is part of a transit signal, otherwise it is 0. We assume that the target $t_j$ depends on all flux values through $x_j$, which is dependent on its neighbours $x_{j\pm1}$, which are dependent on their neighbours, and so on. This can be visualized by:

\begin{figure}[H]
\centering
  \tikz{
% nodes
%  \node[obs]
 \node (xa) {$\dots$};%
 \node[obs,right=of xa] (xb) {$x_{j-1}$};%
 \node[obs,right=of xb] (xc) {$x_{j}$};
 \node[obs,right=of xc] (xd) {$x_{j+1}$};
 \node[right=of xd] (xe) {$\dots$};%
 
 \node[latent,below=of xb] (tb) {$t_{j-1}$};
 \node[latent,below=of xc] (tc) {$t_{j}$};
 \node[latent,below=of xd] (td) {$t_{j+1}$};

\edge [-] {xb} {xa, xc};
\edge [-] {xd} {xc, xe};
\edge {xb} {tb};
\edge {xc} {tc};
\edge {xd} {td};
  }
\end{figure}
\noindent Since we observe $X$, the targets become separated in the graphical model. Therefore we model them as independent, given $X$, i.e.:
\begin{equation}
    p(T|X) = p(t_1,\dots,t_N|X) = \prod_{j=1}^N p(t_j|X)
\end{equation}
In other words, we assume that knowing the target $t_{k}$ at time step $k \neq j$ in addition to the flux values $X$, would not make a difference for the probability of observing $t_j = 1$, compared to only knowing $X$. Therefore, the only information we feed to the network at each time step is $x_j$.

Our network outputs a value $y_j \in (0,1)$ for each $x_j$, using the function $Y = f(X, W)$, where $W$ represents the weights of the network. $Y$ represents the collection of outputs $\{y_j\}_{j=1}^N$. We interpret $y_j$ as the conditional probability $y_j = p(t_j=1|X,W)$, such that $p(t_j=0|X,W)$ is given by $1-y_j$. The conditional distribution of targets at time step $j$ is then given by:
\begin{equation}
    p(t_j|X,W) = y_j^{t_j} (1-y_j)^{1-t_j}
\end{equation}

\noindent Given a set of independent observations $\{T_i\}_{i=1}^M$ for light curve segments $\{X_i\}_{i=1}^M$, we aim to maximize the likelihood $p(T|X,W)$ by updating the weights $W$ of the network. Equivalently, we can maximize the log likelihood, which is given by:
\begin{align}
    \log p(T|X,W) &= \sum_{i=1}^M \log  p(T_i|X_i,W) \\
    &= \sum_{i=1}^M \log \prod_{j=1}^N p(t_{ij}|X_i,W) \\
    &= \sum_{i=1}^M \sum_{j=1}^N \log [ y_{ij}^{t_{ij}} (1-y_{ij})^{1-t_{ij}} ] \\
    &= \sum_{i=1}^M \sum_{j=1}^N t_{ij} \log y_{ij} + (1-t_{ij}) \log(1-y_{ij}),
\end{align}

\noindent where $t_{ij}$ represents the target at time step $j$ in light curve $i$ and $y_{ij} = y(X_i,W)_j$.

To define our loss function, which we aim to minimize, we take the negative of $\log p(T|X,W)$, averaged over the number of samples $M$ and the number of data points per sample $N$, i.e. for a single light curve segment $i$ we have the loss:
\begin{equation}
    \label{eq:bce_loss}
    \mathcal{L}_{\text{BCE},i} = \frac{1}{N}\sum^{N}_{j=1} l_{ij} + (1 - t_{ij}) \log (1 - y_{ij}),
\end{equation}

\noindent where we used $l_{ij} = t_{ij} \log( y_{ij})$, and the subscript BCE because this is known as the binary cross-entropy loss.
Using this loss function for a batched training with $M$ light curve segments per batch, we get:
\begin{equation}
    \mathcal{L}_{\text{BCE}} = \frac{1}{M} \sum_{i=1}^M \mathcal{L}_{\text{BCE},i}.
\end{equation}

