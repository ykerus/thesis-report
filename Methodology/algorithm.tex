
\section{Detection algorithm}

\red{[TODO: highlight that the RNN alone is ready to be used for monotransit detection and is very efficient]}. The RNN can handle arbitrary input sizes, but will only learn local features of a light curve during training, i.e. which data point belongs to a transit signal and which does not. Parameters such as orbital period $P$ of a planet, are ignored by the network. If the network is applied to a full-length light curve after training, it might run into multiple transit events from the same planet. In the task of detecting potential exoplanets, it is important to determine the periodicity of the signal if possible, as this in combination with the epoch $t_0$ allows follow-up research to predict when the next transit events will occur. Furthermore, restricting our search to transit signals that repeat a certain number of times during the span of observations would help in reducing the number of false positive detections.

Following \cite{pearson2018searching}, we refer to all the network outputs $y_i$ over time steps $i$ in a light curve as the probability time series (PTS), while keeping in mind that the name ``probability'' here is not in place correctly. The task of our transit detection algorithm is to determine the epoch and period ($t_0$, $P$) of candidate exoplanets, given the PTS produced by the RNN. Since the PTS only contains the response of the RNN to potential transit signals, each peak in this series could indicate a transit event. To find periodicities, we adopt two different approaches, which are described in the following subsections. 

\subsection{Evaluating peak distances in RNN outputs}
\label{sec:alg_peaks}

As proposed by \cite{pearson2018searching}, one approach to obtain periodicities of potential signals is to evaluate the distances between peaks in the PTS. For this, we use a Gaussian-smoothed PTS, so that peaks are not accidentally counted twice due to noise. We define a peak as a series of values above a certain peak threshold. After having identified the peaks in the PTS, the central time of each event is determined by dividing the peak's area under the curve in two, and taking the time step at the border between the two halves. One could also simply use the time step at the center of the peak, but this was found to produce less accurate results. 

Subsequently, each combination of peaks is inspected for consistent timing. For example, a combination of five peaks could be evaluated and with an average distance of 4 days in the PTS. If the individual distances between subsequent peaks differ for some peaks more than a few hours, the combination of peaks is considered to be inconsistent and therefore rejected. If a combination is not rejected, it is added to a list of candidates.

The list of candidates is then iterated through in order to find the best candidate. For each candidate signal, the estimated transit duration is determined by the median duration of each peak, the estimated period by the median distance between each peak, and $t_0$ by the time of the first peak. For each candidate, the score is defined by sum over all of the peak's maxima, weighted down by the number of peaks that were found. This weighting down is to avoid the algorithm to have a bias towards detecting mainly small period candidates over longer period candidates. If $n$ is the number of peaks under consideration and $s$ the sum over all the peak's maxima, then we define the candidate's score to be given by $s / \sqrt{n}$.

The highest scoring candidate is further analysed, mainly to assess whether single events were missed by the algorithm or the RNN. This last step is added because a few peaks in the PTS might have just not made it above the peak threshold, resulting in the rejection of nearly complete combinations of peaks. The process goes as follows: we iterate over harmonics $h = 1, 2, \dots$, for as long as $P/h > P_{\text{min}}$, where $P$ is the estimated period of the candidate planet and $P_{\text{min}}$ the minimum period to search for. For each $h$ we effectively evaluate a different number $n_h$ of potential transit events, and $s$ now becomes the sum $s_h$ over each maximum value within the estimated transit duration of the candidate signal for the given period $P/h$. The score for harmonic $h$ then becomes $s_h / \sqrt{n_h}$. If $s_h / \sqrt{n_h} > s / \sqrt{n}$, then the candidate's estimated period is adjusted to $P/h$ and the epoch $t_0$ is adjusted accordingly.

\red{[TODO: describe how multiple planets in a single light curve are searched for more or less simultaneously]} \red{[TODO: describe how matching of signal representations is included in the algorithm, which makes use of the representation RNN]}

\subsection{Folding RNN outputs over trial periods}
\label{sec:alg_folding}

Although the algorithm described in Section \ref{sec:alg_peaks} has its benefits, it also has its downsides. The benefit of the approach is that we let the RNN guide us where to search for periodicities. This makes it a very efficient method. However, if the RNN fails to detect one or two transit events in a light curve, or has a response that is not large enough to be picked up by this algorithm, then the algorithm might fail. For this reason, we adopt a more classical approach in addition to the algorithm described in the previous section. 

The approach we take in this algorithm is that of folding the PTS over many trial periods, each time defining a candidate score at each point in the resulting phase curve. The trial periods are chosen such that the data points in the folded curve exactly coincide. In other words, the trial periods are exactly multiples of two minutes in case the input light curve is also of two minute cadence. This allows us to directly sum the PTS values of each set of points coinciding in the phase curve. This sum has the same meaning as $s$ in the previous section, and the number $n$ of transit events under consideration is in this case given by the number of data points that we take the sum of. For each trial period, we thus get a set of different scores $s/\sqrt{n}$, corresponding to different epochs $t_0$. For each trial period we store the maximum score and the corresponding epoch $t_0$. 

We then get a spectrum of scores against different trial periods. This spectrum is evaluated for peaks, where the highest peak in the spectrum corresponds to the highest scoring planet candidate with corresponding period $P$ and epoch $t_0$.

\red{[TODO: describe how multiple planets in a single light curve are searched for iteratively]}
