
\section{Detection algorithm}
\label{sec:algorithm}

In case the search for signals is directed towards monotransits, the network outputs for a given input light curve can directly be used for detection. For example, one can set a threshold on peaks in the PTS, which can be considered as candidate detections. In case we wish to search for repeating signals, the RNN outputs alone are not enough. In order to compare the performance of an RNN-based detection algorithm with conventional methods such as BLS, we also need to determine the period and epoch of the signal. Two different algorithms are tested to do so, which only make use of the RNN outputs, i.e. the PTS, for a given input light curve. 

\subsection{PTS-Peak}
\label{sec:pts-peak}
As proposed by \cite{pearson2018searching}, one could use the distances between peaks in the PTS to determine the period of a repeating signal. Here, we implement this idea and refer to the algorithm as PTS-Peak. In order for this algorithm to work well, we need to take into account the presence of potential false detections in the PTS, or multiple detections of signals belonging to different planets. Therefore we adopt several rules and filtering steps to ensure consistency between matched peaks and resulting parameter estimates. The algorithm used is as follows:

\todo{algorithm description}

 
\subsection{PTS-Fold}
\label{sec:pts-fold}

It could be that individual transit events are missed by the RNN, or that peaks in the PTS are too weak to be taken into account by PTS-Peak. To be less dependent on distinguishable peaks in the PTS, but more on overall response to transit signals, we define another algorithm which we refer to as PTS-Fold. This algorithm is similar to many detection algorithms (e.g. phase dispersion minimization or BLS) in that it folds the input time series over a set of trial periods. However, whereas other algorithms fold the raw light curve over trial periods, PTS-Fold only folds the PTS. Since each value the PTS indicates the extend to which a transit signal might be present at the corresponding time step, we can efficiently compute a candidate detection score for repeating signals by aggregating overlapping data points in the folded PTS. For clarification, we describe the algorithm in steps in the following:

\todo{algorithm description}


