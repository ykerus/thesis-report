
In a series of experiments using simualted data, it was shown how RNNs could be applied for the task of transit signal detection in light curves. The proposed method does not require input light curves to be detrended, nor does it require pre-defined transit models for search. The only requirement is a data set of light curve segments with known transit signals, which is used for training the network. The outputs of the network at each time step can be used for detecting both monotransits, as well as repeating transit signals. The way the outputs can best be used to define a detection criterion should be further investigated future research, but this work showed that two relatively simple algorithms can lead to competative results compared to the BLS algorithm in certain cases. These cases were visually identified to be mostly light curves with noise patterns at time scales of the duration of the transit signals. Based on our experiments, we also conclude that the RNN-based detection algorithm has most potential in cases where the orbital period of the transiting exoplanet is large. Most detection algorithm depend on the periodicity of a signal, which the RNN does not as it is trained on individual transit examples. Especially in the search for monotransits this showed to be beneficial, since the RNN was able to detect more planets in total and at higher precision than a box-fitting algorithm. 

However, the main result of this work is not that there was a slight winner in each of the experiments. In fact, we found both the box-fitting algorithms and the RNN-based algorithms to detect exoplanets that the other methods were not able to detect. From the 2500 monotransits, the RNN detected 298 planets that the box-fitting algorithm was not able to detect at 0.5 precision, which in turn detected 74 planets that the RNN was not able to detect. In the task of retrieving repeating transit signals from 2500 planets, the RNN-based algorithm detected 49 planets that BLS was not able to detect, which in turn detected 224 planets that the RNN-based algorithm was not able to detect. In other words, the methods performed complementary. For exoplanet science, this is an important result, as the RNN might open the door to detecting exoplanets in real-world data previously missed by commonly used detection methods. Future will tell, as this method has yet to be applied to real-world light curves. Even though these may only include a handful of exoplanets, with every new detection our view of the universe, and our place therein, becomes more complete.