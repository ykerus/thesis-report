
In a series of experiments using simulated data, we evaluated how recurrent neural networks (RNNs) could be applied for the task of transit detection in light curves, and how they compare to existing approaches. 

To answer the question of how RNNs could be applied, we evaluated different data preprocessing steps, network architectures and training schemes. Our results suggested that simple gap-filling approaches such as linear interpolation or zero-filling can effectively and efficiently be used to deal with missing data. However, more research is needed in order to establish what method to deal with missing data works best in this task. Detrending or scaling light curves, apart from median scaling, was shown not to be necessary. However, detrending did lead to better results, especially if short-scale patterns were kept in tact. We therefore recommend to use the RNN in combination with a high-pass filter applied to the light curve, which only removes the largest trends.  Throughout this research, we found the most stable and best performing network to be a bidirectional GRU RNN with a single recurrent layer followed by two fully connected layers. This network improved over the more complex LSTM, and multi-layered RNN architectures. The network was trained to classify individual data points in light curve segments as signal or non-signal. During training, we found that a small positive weight ($>$1) applied to all data points belonging to transit signals led to a higher average precision (AP) and allowed for setting more intuitive classification thresholds. Lastly, it was shown that a transit-specific weight can be adopted to balance the focus of the network during training, which also led to an improved AP in some cases.

Two RNN-based algorithms were compared to box-fitting algorithms in the task of detecting transit signals in full-length light curves. These algorithms were referred to as PTS-Fold and PTS-Peak. Although PTS-Peak was at least an order of magnitude more efficient than PTS-Fold, PTS-Fold consistently detected more planets than PTS-Peak. However, the BLS algorithm from \cite{kovacs2002box} still outperformed both algorithms in terms of the number of the detected planets. In the case of monotransits, i.e. transit signals which only occur once in a light curve, we found the RNN to outperform the box-fitting algorithm from \cite{foreman2016population}. In that case, the RNN outputs can directly be used for detection, and light curves can be processed very efficiently ($>$50 light curves with $\sim$ 20000 data points per second). Overall, we found that the fewer the transits, the more potential the RNN has of improving over box-fitting algorithms.

% The proposed method does not require input light curves to be detrended, nor does it require pre-defined transit models for search. The only requirement is a data set of light curve segments with known transit signals, which is used for training the network. The outputs of the network at each time step can be used for detecting both monotransits, as well as repeating transit signals. The way the outputs can best be used to define a detection criterion should be further investigated future research, but our research showed that two relatively simple algorithms can lead to competitive results compared to the BLS algorithm in certain cases. These cases were visually identified to be mostly light curves with noise patterns at time scales of the duration of the transit signals. Based on our experiments, we also conclude that the RNN-based detection algorithm has most potential in cases where the orbital period of the transiting exoplanet is large. Most detection algorithm depend on the periodicity of a signal, which the RNN does not as it is trained on individual transit examples. Especially in the search for monotransits this showed to be beneficial, since the RNN was able to detect more planets in total and at higher precision than a box-fitting algorithm. 

However, the main result of our research is not that there was a slight winner in each of the experiments. In fact, we found both the box-fitting algorithms and the RNN-based algorithms to detect exoplanets that the other methods were not able to detect. From the 2500 monotransits, the RNN detected 298 planets which the box-fitting algorithm was not able to detect at 0.5 precision, which in turn detected 74 planets which the RNN was not able to detect. In the task of retrieving repeating transit signals from 2500 planets, the RNN-based algorithm detected 49 planets which BLS was not able to detect, which in turn detected 224 planets which the RNN-based algorithm was not able to detect. In other words, the methods performed complementary. For exoplanet science, this is an important result, as the RNN might open the door to detecting exoplanets in real-world data previously missed by commonly used detection methods. Future will tell, as this method has yet to be applied to real-world light curves. Even though these may only include a handful of undiscovered exoplanets, with every new detection our view of the universe, and our place therein, becomes more complete.