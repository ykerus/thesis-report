In the following, we first investigate how RNN could best be applied for our task, after which we compare our RNN-based algorithms to existing approaches. The first section of this chapter is concerned with preprocessing of the data that is fed to the RNN, where we evaluate different scaling and gap handling approaches and their effect on the performance of the RNN in the given task. Often in classification tasks, the accuracy is used as a measure of the model's performance. However, since we are dealing with an imbalanced data set, the accuracy might give a wrong impression. This is because classifying each data point as non-signal would already result in high accuracy. Therefore we primarily look at the precision and recall, which are given by:
\begin{align}
    \text{precision} &= \frac{tp}{tp + fp}\\
    \text{recall} &= \frac{tp}{tp + fn},
\end{align}
where $tp$ is the number of true positive classifications, $fp$ the number of false positives and $fn$ the number of false negatives.  However, accuracy, precision, recall and related metrics such as F1-score are dependent on a classification threshold, which defines when a data point is classified as signal ($y_{ij} \geq \text{threshold}$) or non-signal ($y_{ij} < \text{threshold}$), For this reason, we use area under the precision-recall (PR) curve as a measure that is independent of the threshold. The area under the PR-curve is also known as average precision (AP). 

Subsequently, we compare different models and training schemes in Section \ref{sec:models}. This is to motivate our architecture choice for the RNN used in the detection algorithm, and to assess whether RNNs can compete with the more commonly used CNN. In the sections that follow, i.e. \ref{sec:monos}, \ref{sec:singles} and \ref{sec:multis} we compare our RNN-based transit detection algorithms with box-fitting algorithms in real-world problems. These include, respectively, the retrieval of monotransits, the retrieval of single planets with repeating transit signals, and the retrieval of potentially multiple planets with repeating transit signals in a light curve. Finally, we take a closer look at a few success and failure cases in Section \ref{sec:cases}.
