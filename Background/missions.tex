
\section{Exoplanet-hunting missions}
\label{sec:missions}

Since the first exoplanet discovery \citep{wolszczan1992planetary}, the rate at which exoplanets are discovered has grown exponentially. Ground-based surveys such as WASP \citep{pollacco2006wasp}, HAT \citep{bakos2004wide}, TRAPPIST \citep{jehin2011trappist} and KELT \citep{pepper2007kilodegree} are responsible for the discovery of several exoplanets. However, the major turning point was when the search for transiting exoplanets was taken into space.

CoRoT \citep{auvergne2009corot} was the first space-based mission designed to search for transiting exoplanets and added 33 exoplanets to the list. Kepler \citep{koch2010kepler} and its continued mission K2 \citep{howell2014k2} hold the discovery of over half of the known exoplanets, i.e. over $2500$.
Kepler observed a small patch of the sky and monitored over 500,000 stars during its mission. Most targets were observed with an observation interval of 30 minutes, i.e. a 30-minute cadence. TESS \citep{ricker2014transiting}, Kepler’s successor, monitors brighter and closer stars in a full-sky survey. After each 27.4-day sector, the telescope points at a different patch of sky. Midway each sector, the spacecraft transmits data back to Earth, during which no new data is gathered. With over 200,000 stars observed and ``only’’ 148 exoplanets discovered, hundreds to thousands discoveries more are expected \citep{sullivan2015transiting}. The amount of data in the field will only increase more rapidly, as future telescopes will get bigger and better.
Planned for launch in 2026, PLATO will monitor the brightness of around a quarter million stars for the detection and characterization of transiting exoplanets \citep{rauer2021plato}.

