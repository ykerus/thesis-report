
\section{Identification methods}

While the focus of this thesis lies on detection, a brief overview of some of the existing work on identification is given both for completeness, and to give the reader a better sense what machine learning methods have been applied in this field. 

The identification of candidate signals is in general a binary classification problem, i.e. signal or non-signal. For this reason, the 1D-CNN of \cite{pearson2018searching} can be be considered an identification model. In fact, \cite{pearson2018searching} provide a comparison between several models for the task of identification (even though it is referred to as ``detection''). These models include a support vector machine (SVM), multi-layer perceptron (MLP), a box function as matched filter, an MLP that takes the wavelet decomposition of the light curve as input, and their 1D-CNN. This work inspired \cite{shallue2018identifying} to develop \textit{Astronet}, a 1D-CNN optimized for transit identification. Several papers followed that describe applications of or variations to \textit{Astronet}. For example, with and without alterations to the model, \cite{dattilo2019identifying} applied \textit{Astronet} to K2 data, \cite{osborn2020rapid} and \cite{yu2019identifying} applied the model to TESS data \cite{ansdell2018scientific} added scientific domain knowledge to the model such as centroid data and stellar parameters. \cite{koning2019reducing} proposed methods to reduce the network complexity. 

Instead of supervised learning, \cite{armstrong2016transit} made use of self-organizing-maps (SOMs), which are trained to cluster the data in an unsupervised manner. Other methods which do not rely on neural networks, but on decision trees and random forests, are \textit{Autovetter} and \textit{Robovetter} \citep{catanzarite2015autovetter, coughlin2017planet}. \cite{thompson2015machine} identify transit signals using a method based on k-nearest neighbours.

Several more works exist that compare different models, e.g. \cite{schanche2019machine} compare for example an SVM, random forest classifier, k-nearest neighbours and logistic regression. Lastly, \cite{jaratransiting} compare a wide range of identification models. In addition, they extend the models to take as input the wavelet decomposition, or multi-resolution analysis (MRA), coefficitions instead of the raw light curve, and motivate how MRA can improve transit identification.