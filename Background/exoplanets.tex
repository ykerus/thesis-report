
\section{Exoplanets}

Most people are familiar with the planets in our solar system, e.g. Venus, Mars, Jupiter. What these planets have in common is that they all orbit the same star, that is our Sun. Exoplanets might be similar to the planets we know well, but they orbit different stars, which makes them hard to find and difficult to study.

\subsection{A brief history of exoplanet science}
For a long time, Earth was one of the few planets known to human kind, as part of the only planetary system known. However, billions of stars exist within our galaxy alone, each potentially with their own orbiting exoplanets. [TODO: list discoveries] Later on, more exoplanets were discovered, and thus the list of known planetary systems was extended. 


\subsection{Discovery methods}
\red{[TODO]}
\subsubsection{Transit method}
\red{[TODO]}
\subsubsection{Transit timing variations}
\red{[TODO]}
\subsubsection{Radial velocity}
\red{[TODO]}
\subsubsection{Direct imaging}
\red{[TODO]}
\subsubsection{Microlensing}
\red{[TODO]}

\subsection{Exoplanet-hunting missions}
\red{[TODO]}
\subsubsection{Ground-based surveys}
\red{[TODO]}
\subsubsection{CoRoT}
\red{[TODO]}
\subsubsection{Kepler and K2}
\red{[TODO]}
\subsubsection{TESS}
\red{[TODO]}
\subsubsection{PLATO}
\red{[TODO]}