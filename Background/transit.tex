
\section{Transit method}
\label{sec:transit}


From all exoplanet discovery methods, the transit method has most exoplanet discoveries on its name. At the time of writing, over 75\% (3343 in total) of the known exoplanets (4424 in total) were discovered using this method. Of the other methods, the radial velocity method is the most successful and covers about 20\% of the exoplanet discoveries. The radial velocity method works by measuring the shifts in the spectrum of a star which are due to the gravitational pull of an orbiting planet. While this used to be the prominent method for exoplanet discovery, nowadays it is mostly used to complement the transit method as a means of confirming exoplanet candidates, or characterizing planetary systems. Anther method closely related to the transit method uses transit timing variations (TTVs) of transiting planets to infer the presence of an additional planet in the same system. TTVs are deviations from the expected periodicity of a signal, which could be caused by a disturbing planet in a different orbit around the same star. Other methods to exoplanet discovery include for example gravitational microlensing and direct imaging, but for the purpose of this thesis we do not go into depth for these methods.

Exoplanets are far away and form a compact system together with their host star. This makes direct imaging difficult. Therefore, most methods rely on indirect measurements of potential exoplanets. This is also the case for the transit method, because only the brightness of a star is monitored over time. Dips in the observed brightness can indicate an exoplanet passing in front of the stellar disk. A requirement for this method is therefore that the system is seen edge-on.

From the shape of the transit signal we can learn several parameters about the planet and its host star. For example, the depth of the signal relates to the planet size ($\text{depth} = R_{\text{planet}}^2/ R_{\text{star}}^2$). The duration relates to the period of the signal, which in turn relates to the distance of the planet from its host star.

\todo{explain phase folded light curve and how it increases SNR}

\todo{describe ingress, egress, mid-transit, etc.}

\todo{\begin{equation}
    \label{eq:kepler}
    P^2 = \frac{4 \pi^2}{GM}  a^3.
\end{equation}}