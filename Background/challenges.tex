
\section{Challenges in detecting transit signals}
\label{sec:challenges}

In the task of transit detection, the presence of an exoplanet is not known beforehand, 
so we need to search the light curves for potential signals. However, many noise sources hinder the search and could trigger false detections. This and other challenges are described in the following subsections.

\subsection{Stellar variability}
Stars do not emit photons at a constant rate. A star's surface is dominated by cell-like structures called granules. Energy transport within the star causes hotter and brighter cells of plasma to rise, and cooler and dimmer plasma to descent. This granulation effect leaves the stellar surface constantly changing, and results in fluctuations in the star’s light curve. Granulation has timescales of about a half hour up to several days \citep{kallinger2014connection}. At larger time scales, starspots could leave their imprint on a light curve. These visually dark regions on the star's surface are relatively low in temperature, and thus decrease the flux. As the star rotates, the starspot could come in and out of view, which results in large scale fluctuations in the star’s light curve. This effect is called rotation modulation. Additionally, other phenomena could be at play such as stellar oscillations or sudden outbursts of energy called flares. In short, a light curve may exhibit quasi-periodic patterns caused by complex mechanisms.

\subsection{Instrumental and photon noise}
All data used in this work is assumed to be collected by space-based observatories. Although space offers a good environment for astronomical observations, the instruments still suffer from imperfections. For example, temperature changes in the spacecraft might alter the pointing of the telescope, which could cause jumps or drifts in the observed flux from stars in the field of view. Incorporating information on the telescope’s pointing, or centroids, in addition to the flux values seems beneficial in some exoplanet-related tasks \todo{cite centroid examples}.  Stray light, for example reflected off the moon, might also influence the quality of observations. Poor quality data may be flagged by the spacecraft’s pipeline, which could then be filtered out. However, this leaves missing data points in the light curve, which might pose a problem to certain processing algorithms. 
Another source of noise, although not due to instrument imperfections, is photon noise. This noise originates from the arrival of different amounts of photons at the detector between different observations. In general, this type of noise can be seen as the time independent Gaussian distributed noise.

\subsection{Transit signal shapes}
\label{sec:transit_shapes}

The basic shape of a transit signal is determined by, among others, the planet’s orbital period, radius relative to its host star, distance from its host star, and inclination and eccentricity of its orbit. Each parameter affects the shape of a transit differently. The stellar surface also plays a role in the shape of the signal. The limbs of the star seem darker than its central region. This is because at the central region, we look deeper into the star where temperatures are higher. When a planet passes in front of the star, it therefore blocks a varying portion of the star’s total emitted light. For this reason, a box-shaped transit model is not accurate. More realistic transit models take this limb darkening effect into account. Other phenomena can also cause transit depths to vary over time, for example when a planet passes in front of starspots during its transit. Although it is not the task of the transit detection algorithm to determine all the parameters that cause a certain transit shape to be observed, it should be robust against different signal shapes during its search.

\subsection{Weak and sparse signals in big data}

The requirement for the transit to work is that the planetary system is seen edge-on. However, nothing restricts a system from being oriented randomly. For this reason, even if all stars would host exoplanets, a large portion would still go undetected. Moreover, the farther the planet from the host star, the smaller the chance that the planet transits the star. For example, the chance of earth to transit the sun if it were observed from outside the solar system is only 0.47\% \citep{borucki1984photometric}. Even if it does, Earth transits the Sun only once per year and leaves dip of 80 ppm in the Sun’s observed flux for 13 hours. In other words, transit signals are weak and sparse, especially those we are most interested in. The reason why still many exoplanets are discovered by the transit method, is mainly due to large amount of available data. For example, TESS alone collects over 20 GB every day \todo{cite TESS data amount}, which alone could also be a challenge to work with.

\subsection{Astrophysical false positives}
\label{sec:astro_false_pos}

Sometimes, a false positive detection is not due to stellar or spacecraft induced noise, but due to other phenomena. For example, a large portion of the stars exists in a binary configuration. In case an eclipsing binary (EB) system is seen edge-on from Earth, the stars transit each other just like an exoplanet would transit its host star. Generally, EB signals are larger than for exoplanets, because stars are larger than planets. However, if the stars have grazing transits, the signals can become smaller and more similar to exoplanet transits. EB systems lurking in the background of an observed star can also produce disturbing signals. These background EBs (BEBs) are much farther away than the star under consideration, but contribute to the star's observed light curve. The BEB transit signals will thus be smaller than EB signals and could mimic exoplanet transit signals in the light curve of the foreground star. In the detection step, however, it is not the main goal to discern (B)EBs from exoplanet transit signals. Preferably, most EB signals are filtered out from the start, but restricting our detection algorithm too much on what kind of signals it is allowed to pick up, may exclude various interesting signals from being detected. For example, the detection step could return a signal that is in fact caused by a large exocomet or asteroid instead of an exoplanet, which would still be of great astrophysical value to find and study. Furthermore, such detected signals may well be filtered out in the vetting step. For this reason, we exclude astrophysical false positives from this work, and leave them for future research.